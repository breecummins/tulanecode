\documentclass{article}

\usepackage[top=1in,bottom=0.5in,left=0.75in,right=0.75in]{geometry}
\usepackage{amsmath,amssymb,multirow}

\newcommand{\bx}{\mathbf{x}}
\newcommand{\by}{\mathbf{y}}
\newcommand{\bn}{\mathbf{n}}

\begin{document}
\section*{Table of Quadrature Results}

We are solving the Laplacian $\Delta u(\bx) = 0$ on the unit disk $D$ with boundary conditions $g(\theta) = \sin(3\theta)$ on $\partial D$. The solution in boundary integral form is given by:
\[ u(\bx) = \int_{\partial D} f(\by) \frac{\partial}{\partial \bn} G(\by,\bx) \; \mbox{d}\by,  \]
where $f(\by) = f(r,\theta) = 2 \sin(3\theta)$ is the density function associated with the boundary conditions, $\bn$ is the outward pointing normal, and $G(\by,\bx) = \ln(\lvert\by - \bx\rvert)/(2\pi)$ is the free-space Green's function for the Laplacian. The normal derivative of $G$ is given by
\[ \frac{\partial}{\partial \bn} G(\by,\bx) = \frac{(\by - \bx)\cdot\bn}{2\pi\lvert\by - \bx\rvert^2}. \]
The solution within the unit disk is known to be $u(r,\theta) = r^3\sin(3\theta)$, and the solution outside is $u(r,\theta) = -r^{-3}\sin(3\theta)$.

The unit circle is discretized into $N$ points with chords between each pair of points on which to do quadrature. Each chord is divided into $M$ points. The boundary elements are either constant on each chord or they are bilinear ``hats" that span 2 chords, one rise and one fall. Quadrature is performed using the composite trapezoid rule on each chord.  

\begin{table}[h]
	\begin{centering}
\begin{tabular}{|c|c|c|c|c||c|c|c|c|c|}
	\hline
	Basis Element& $N$ & $h$ & Error & Error/$h$ & Basis Element& $N$ & $h$ & Error & Error/$h^2$ \\
	\hline
	 \multirow{8}{*}{Constant}	 & 	4  & 1.4142	&    8.2210e-01	&  0.5813 & \multirow{8}{*}{Hat}  &    4     	&  1.4142	&	1.1575e+00 &  0.5787  \\
							 	 &   8 & 0.7654	&    7.5498e-01 &  0.9864 &                       &    8 		& 0.7654	&	2.7571e-01 &  0.4707   \\
							 	 &  16 & 0.3902	&    1.6655e-02 &  0.0427 &                       &   16 		& 0.3902	&	1.3790e-01 &  0.9058   \\
							 	 &  32 & 0.1960	&    7.5065e-02 &  0.3829 &                       &   32 		& 0.1960	&	6.4301e-02 &  1.6732   \\
							 	 &  64 & 0.0981	&    4.5538e-02 &  0.4640 &                       &   64 		& 0.0981	&	1.1335e-03 &  0.1177   \\
							 	 & 128 & 0.0491	&    2.6521e-02 &  0.5403 &                       &  128 		& 0.0491	&	2.6648e-03 &  1.1061   \\
							 	 & 256 & 0.0245	&    1.6822e-02 &  0.6854 &                       &  256 		& 0.0245	&	1.0026e-03 &  1.6644   \\
							 	 & 512 & 0.0123	&    8.8121e-03 &  0.7181 &                       &  512 		& 0.0123	&	1.0052e-04 &  0.6675   \\
							 	 & 1024 &0.0061	&    4.4558e-03 &  0.7262 &                       &  1024		& 0.0061	&	3.0173e-05 &  0.8014   \\
 \hline   
\end{tabular}
\caption{BEM with the trapezoid rule.  $N$ is the number of chords; $M$ is fixed at $1000$; the error is for the point $1.01*(\cos(0.7),\sin(0.7))$. The spacing between the points $h$ is given by the length of the chord $h = 2\sin(\pi/N)$.}
\end{centering}
\end{table}

When increasing the number of chords using constant functions, we achieve about $O(h)$ convergence. When using hat functions, the order is approximately $h^2$; when dividing by $h$, there is a rapidly decreasing sequence of numbers and when dividing by $h^3$, there is a rapidly increasing sequence. This indicates that even though the values jump around, the convergence is about $h^2$.

\end{document}





























































































































