\documentclass[12pt]{article}
\usepackage{amsmath,amssymb,latexsym}
\textwidth6.5in\oddsidemargin 0.00in 
\parskip 4pt
\parindent 0pt
%-------------------------------------------------------------
% Define some commands to be used
%-------------------------------------------------------------
% DEFINE SOME STYLE PARAMETERS
\newcommand{\bs}{\mbox{\bf s}}
\newcommand{\bx}{\mbox{\bf x}}
\newcommand{\by}{\mbox{\bf y}}
\newcommand{\br}{\mbox{\bf r}}
\newcommand{\bv}{\mbox{\bf v}}
\newcommand{\tv}{\mbox{$\tilde v$}}
\newcommand{\bu}{\mbox{\bf u}}
\newcommand{\tu}{\mbox{$\tilde u$}}
\newcommand{\tw}{\mbox{$\tilde w$}}
\newcommand{\bU}{\mbox{\bf U}}
\newcommand{\FF}{\mbox{\bf F}}
\newcommand{\ff}{\mbox{\bf f}}
\newcommand{\fn}{\mbox{$\ff - (\ff\cdot\bs)\bs$}}
\newcommand{\fs}{\mbox{$(\ff\cdot\bs)$}}
\newcommand{\fd}{\mbox{$\phi_{\delta}$}}
%\newcommand{\fd}{\mbox{$\phi$}}
\newcommand{\Gd}{\mbox{$G_{\delta}$}}
\newcommand{\Bd}{\mbox{$B_{\delta}$}}
%--------------------------------------------------------
\begin{document}

\begin{center}
{\bf\large  Brinkmanlets + Dipoles in 2D}
\end{center}


%%%
\section{Introduction}
%%%
According to Bree's writeup, the singular Brinkmanlet in 2D is
\begin{equation}
\mu \bv(\bx) = \ff H_1(r) + (\ff\cdot\bx)\bx H_2(r)
\end{equation}
where
\begin{eqnarray}
H_1(r) &=&  \left[ \frac{K_1(\lambda r) + \lambda r K_0(\lambda r)}{2\pi \lambda r} - \frac{1}{2\pi\lambda^2 r^2}\right]
= \left( \frac{1}{2\pi\lambda r} \right) \left[ \lambda r K_0(\lambda r) + K_1(\lambda r) - \frac{1}{\lambda r} \right] \\
H_2(r) &=& \left[ -\frac{2K_1(\lambda r) + \lambda r K_0(\lambda r)}{2\pi \lambda r^3} + \frac{2}{2\pi\lambda^2 r^4}\right]
= \left( \frac{1}{2\pi\lambda^2 r^4} \right) \left[ 2- \lambda^2 r^2 K_2(\lambda r)  \right]
\end{eqnarray}

According to me, we have that 
\[
\mu\Delta\bv(\bx) = \ff Q_1(r) + (\ff\cdot\bx)\bx Q_2(r)
\]
where
\begin{eqnarray}
Q_1(r) &=&  \Delta H_1(r) + 2 H_2(r), \\
Q_2(r) &=& \Delta H_2(r) + 4 \frac{H'_2(r)}{r}
\end{eqnarray}
This leads to
\begin{eqnarray}
Q_1(r) &=&  \left(\frac{\lambda}{2\pi r}\right) \left[ \lambda r K_0(\lambda r) +  K_1(\lambda r) \right], \\
Q_2(r) &=& -\left(\frac{\lambda^2}{2\pi r^2}\right)  K_2(\lambda r) 
\end{eqnarray}

So, the Brinkmanlet $+$ dipole velocity is
\[
\mu\bv(\bx) = \ff [ H_1(r) + \alpha Q_1(r)] + (\ff\cdot\bx)\bx [H_2(r) + \alpha Q_2(r)]
\]

If we evaluate it at $r=a$ and set the solution equal to $(U,0)$, we have that $\ff = (f,0)$ and
\begin{eqnarray}
U &=& f [ H_1(a) + \alpha Q_1(a)],  \\
0 &=& H_2(a) + \alpha Q_2(a)
\end{eqnarray}
The second equation gives
\[
\alpha = - \frac{H_2(a)}{Q_2(a)} = \frac{1}{\lambda^2} \left( \frac{2}{\lambda^2 a^2 K_2(\lambda a)} -1\right)
\]
and the first one 
\begin{eqnarray*}
\frac{U}{f}&=&H_1(a) + \alpha Q_1(a) \\
&=&\left( \frac{1}{2\pi\lambda a} \right) \left[ \lambda a K_0(\lambda a) + K_1(\lambda a) - \frac{1}{\lambda a} \right] + \alpha  \left(\frac{\lambda}{2\pi a}\right) \left[ \lambda a K_0(\lambda a) +  K_1(\lambda a) \right] \\
&=& \left( \frac{1}{2\pi\lambda a} \right) \left[ \lambda a K_0(\lambda a) + K_1(\lambda a) - \frac{1}{\lambda a}   -\frac{
 \left[ \lambda a K_0(\lambda a) +2 K_1(\lambda a) - \frac{2}{\lambda a} \right] }{
 \left[ \lambda a K_0(\lambda a) +  2 K_1(\lambda a) \right] }  \left[ \lambda a K_0(\lambda a) +  K_1(\lambda a) \right] \right] \\
 &=& \left(\frac{1}{2\pi\lambda a}\right) \frac{ K_0(\lambda a)}{ \lambda a K_0(\lambda a) + 2 K_1(\lambda a)} 
\end{eqnarray*}
or
\[
f = U \frac{ 2\pi\lambda^2 a^2  K_2(\lambda a)}{K_0(\lambda a)}
\]

To simplify the velocity equation, we use
\begin{eqnarray*}
H_1(r) + \alpha Q_1(r) &=&  \left( \frac{1}{2\pi\lambda r} \right) \left[ \lambda r K_0(\lambda r) + K_1(\lambda r) - \frac{1}{\lambda r} \right] + \alpha \left(\frac{\lambda}{2\pi r}\right) \left[ \lambda r K_0(\lambda r) +  K_1(\lambda r) \right] \\
&=&  \left( \frac{1}{2\pi\lambda r} \right) \left[ (1+\alpha\lambda^2) (\lambda r K_0(\lambda r) + K_1(\lambda r)) - \frac{1}{\lambda r} \right]  \\
&=&  \left( \frac{1}{2\pi\lambda r} \right) \left[ \frac{2(\lambda r K_0(\lambda r) + K_1(\lambda r))}{\lambda^2 a^2 K_2(\lambda a)}  - \frac{1}{\lambda r} \right]
\end{eqnarray*}
which leads to
\[
f ( H_1(r) + \alpha Q_1(r) ) = \frac{U}{K_0(\lambda a)} \left( K_0(\lambda r) + K_2(\lambda r) - \frac{a^2}{r^2}K_2(\lambda a) \right)
\]

Also
\begin{eqnarray*}
H_2(r) + \alpha Q_2(r) &=& \left( \frac{1}{2\pi\lambda^2 r^4} \right) \left[ 2- \lambda^2 r^2 K_2(\lambda r)  \right]
- \alpha \left(\frac{\lambda^2}{2\pi r^2}\right)  K_2(\lambda r) \\
&=&  \frac{2}{2\pi\lambda^2 r^4} - (1+\alpha\lambda^2) \frac{K_2(\lambda r)}{2\pi r^2} \\
&=&  \frac{2}{2\pi\lambda^2 r^4} \left( 1 - \frac{r^2 K_2(\lambda r)}{ a^2 K_2(\lambda a)}  \right)
\end{eqnarray*}
which leads to
\[f r^2 ( H_2(r) + \alpha Q_2(r) ) = \frac{2}{K_0(\lambda a)}\left( \frac{a^2}{r^2} K_2(\lambda a) - K_2(\lambda r) \right)
\]

So the velocity becomes
\begin{eqnarray}
\mu u(\bx) &=& \frac{U}{K_0(\lambda a)} \left( K_0(\lambda r) + K_2(\lambda r) - \frac{a^2}{r^2}K_2(\lambda a) \right) 
+  \frac{2\cos^2\psi}{K_0(\lambda a)}\left( \frac{a^2}{r^2} K_2(\lambda a) - K_2(\lambda r) \right) \\
\mu v(\bx) &=&  \frac{2\sin\psi\cos\psi}{K_0(\lambda a)}\left( \frac{a^2}{r^2} K_2(\lambda a) - K_2(\lambda r) \right)
\end{eqnarray}



\end{document}







