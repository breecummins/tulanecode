\documentclass[12pt]{article}

\usepackage[top=0.75in,bottom=0.75in,left=0.5in,right=0.5in]{geometry}
\usepackage{amsmath,amssymb,multirow,graphicx,wrapfig}

\newcommand{\bee}[1]{\begin{equation} #1 \end{equation}}
\newcommand{\baa}[1]{\begin{eqnarray} #1 \end{eqnarray}}
\newcommand{\bees}[1]{\begin{equation*} #1 \end{equation*}}
\newcommand{\baas}[1]{\begin{eqnarray*} #1 \end{eqnarray*}}

\newcommand{\pd}[2]{\ensuremath{\frac{\partial #1}{\partial #2}}}
\newcommand{\dd}[2]{\ensuremath{\frac{d #1}{d #2}}}

\newcommand{\bu}{{\mathbf u}}
\newcommand{\bx}{{\mathbf x}}
\newcommand{\ff}{{\mathbf f}}
\newcommand{\pe}{\phi_\epsilon}
\newcommand{\Ge}{G_\epsilon}
\newcommand{\Be}{B_\epsilon}
\newcommand{\Bd}{B_\delta}
\newcommand{\Fd}{F_\delta}
\newcommand{\eps}{\epsilon}



\title{Showing divergence free velocity fields from a Stokeslet}
\author{Bree Cummins}

\begin{document}
\maketitle

We have that the velocity from a regularized Stokeslet (with fluid viscosity $\mu =1$) is
\bees{
\bu =(\ff \cdot \nabla)\nabla\Be(r) - \ff\nabla^2\Be(r), 
}
where $r = |\bx|$ and the radially symmetric function $\Be$ varies with the choice of blob function. The assumption of radial symmetry allows us to make some simplifications. In the following, $\partial_i$ means take the derivative with respect to the $i$-th Cartesian coordinate, and $n$ is the dimension of the domain.
\baas{
\nabla\Be &=& \frac{\Be'(r)}{r}\bx \\
\left(\left( \ff \cdot \nabla \right) \nabla\Be\right)_j &=& \sum_{i=1}^n f_i\partial_i\left( \frac{\Be'(r)}{r} x_j\right) \\
&=& \sum_{i=1}^n \frac{\Be'(r)}{r} f_i \delta_{ij} + \sum_{i=1}^n x_j f_i\partial_i \frac{\Be'(r)}{r} \\
&=&  \frac{\Be'(r)}{r} f_j + \sum_{i=1}^n x_j \frac{f_i x_i}{r} \left( \frac{\Be''(r)}{r} - \frac{\Be'(r)}{r^2} \right) \\
\Rightarrow \left( \ff \cdot \nabla \right)\nabla\Be &=& \frac{\Be'(r)}{r} \ff + \left( \frac{\Be''(r)}{r^2} - \frac{\Be'(r)}{r^3} \right)(\ff \cdot \bx) \bx.
}
Then we may write
\baa{
\bu &=& \left(\frac{\Be'(r)}{r}-\nabla^2\Be(r)\right) \ff + \left( \frac{\Be''(r)}{r^2} - \frac{\Be'(r)}{r^3} \right)(\ff \cdot \bx) \bx \nonumber\\
&=& H_1(r) \ff + H_2(r) (\ff \cdot \bx) \bx. \label{eq1}
}
The Laplacian in $H_1$ introduces different dependencies on $\Be'(r)$ in two and three dimensions. In two dimensions,
\baa{
H_1^{2D} &=& \frac{\Be'(r)}{r}-\nabla^2\Be \nonumber\\
&=& \frac{\Be'(r)}{r} - \frac{1}{r}\left( r\Be'(r)\right)' \nonumber\\
&=& \frac{\Be'(r)}{r} - \frac{1}{r}\left( \Be'(r)+r \Be''(r)\right) \nonumber\\
&=& -\Be''(r). \label{H12D}
}
Additionally, in two dimensions we add a constant to $H_1$, to remove a constant flow induced everywhere by the Stokeslet. This is equivalent to solving for a shifted velocity field. Since the divergence of a constant times the force, $c\ff$, is zero, we ignore it in the following work.

In three dimensions, the Laplacian has a different form, and consequently, so does $H_1$:
\baa{
H_1^{3D} &=& \frac{\Be'(r)}{r}-\nabla^2\Be \nonumber\\
&=& \frac{\Be'(r)}{r} - \frac{1}{r^2}\left( r^2\Be'(r)\right)' \nonumber\\
&=& \frac{\Be'(r)}{r} - \frac{1}{r^2}\left( 2r\Be'(r)- r^2 \Be''(r)\right) \nonumber\\
&=& \frac{\Be'(r)}{r} - \frac{2\Be'(r)}{r} - \Be''(r) \nonumber\\
&=&  - \Be''(r) - \frac{\Be'(r)}{r}. \label{H22D}
}

We take the divergence (in Cartesian coordinates) of Eq.~\eqref{eq1} in $n$ dimensions:
\baas{
\nabla \cdot \bu &=& \sum_{i=1}^n \partial_i(H_1(r)f_i)  + \partial_i(H_2(r)(\ff \cdot \bx)x_i)\\
&=& \sum_{i=1}^n \frac{H_1'(r)}{r}f_ix_i + \sum_{i=1}^n \frac{H_2'(r)}{r}(\ff \cdot \bx)x_i^2 + \sum_{i=1}^n H_2(r)(\ff \cdot \bx) + \sum_{i=1}^n H_2(r)f_ix_i\\
&=& \frac{H_1'(r)}{r}(\ff \cdot \bx) + \frac{H_2'(r)}{r}(\ff \cdot \bx)r^2 + nH_2(r)(\ff \cdot \bx)+ H_2(r)(\ff \cdot \bx) \\
&=& (\ff \cdot \bx) \left( \frac{H_1'(r)}{r} + r H_2'(r) + (n+1) H_2(r)\right).
}
In order for $\bu$ to be divergence free, we require that $\frac{H_1'(r)}{r} + r H_2'(r) + (n+1) H_2(r) = 0$. In both two and three dimensions, we have that 
\baas{
H_2'(r) &=& \pd{}{r}\left( \frac{\Be''(r)}{r^2} - \frac{\Be'(r)}{r^3} \right) \\
&=& \frac{\Be'''(r)}{r^2} -\frac{2\Be''(r)}{r^3} - \frac{\Be''(r)}{r^3} + \frac{3\Be'(r)}{r^4} \\
\Rightarrow r H_2'(r) &=& \frac{\Be'''(r)}{r} -\frac{3\Be''(r)}{r^2} + \frac{3\Be'(r)}{r^3}.
}
In two dimensions,
\baas{
\frac{(H_1^{2D})'(r)}{r} &=& -\frac{\Be'''(r)}{r},
}
so that 
\baas{
\frac{H_1'(r)}{r} + r H_2'(r) + (n+1) H_2(r) &=& -\frac{\Be'''(r)}{r} + \frac{\Be'''(r)}{r} -\frac{3\Be''(r)}{r^2} + \frac{3\Be'(r)}{r^3} + \frac{3\Be''(r)}{r^2} - \frac{3\Be'(r)}{r^3}\\
&=& 0,
}
yielding a divergence free velocity field as desired.
In three dimensions,
\baas{
\frac{(H_1^{3D})'(r)}{r} &=& \frac{1}{r}\left( - \Be'''(r) - \frac{\Be''(r)}{r} + \frac{\Be'(r)}{r^2}\right) \\
&=& - \frac{\Be'''(r)}{r} - \frac{\Be''(r)}{r^2} + \frac{\Be'(r)}{r^3},
}
so that 
\baas{
\frac{H_1'(r)}{r} + r H_2'(r) + (n+1) H_2(r) &=& -\frac{\Be'''(r)}{r} - \frac{\Be''(r)}{r^2} + \frac{\Be'(r)}{r^3} + \frac{\Be'''(r)}{r} -\frac{3\Be''(r)}{r^2} + \frac{3\Be'(r)}{r^3} \\
&& + \frac{4\Be''(r)}{r^2} - \frac{4\Be'(r)}{r^3}\\
&=& 0,
}
also yielding a divergence free velocity. This is the expected result, since the solution for $\bu$ is constructed using the incompressibility condition. 

However, instead of starting with a blob function and deriving $\Be$, one could replace $B'(r) = -1/8\pi$ (the derivative of the biharmonic function in the singular case) with a regularizing function $\Bd'(r) = -(1/8\pi)\Fd(r)$. By defining $H_1$ and $H_2$ using $\Bd'(r)$ instead of $\Be'(r)$ in Eqs.~\eqref{eq1}, \eqref{H12D}, and \eqref{H22D}, we have just shown that a divergence free velocity would result. One could then back-solve to discover the associated blob function. This is what Karin did in the Brinkman paper.




\end{document}

















